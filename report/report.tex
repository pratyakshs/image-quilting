\documentclass[12pt,a4paper]{report}
\usepackage[utf8]{inputenc}
\usepackage{amsmath}
\usepackage{amsfonts}
\usepackage{amssymb}
\usepackage{graphicx}

\title{\textbf{Image Quilting for Texture Synthesis and Transfer}}
\author{
		\textbf{Project Report}\\
		\textbf{CS 663: Digital Image Processing}\\
		\\
		by\\
        \textbf{Pratyaksh Sharma}\\
		\textbf{Manik Dhar}\\
		\textbf{Ranveer Aggarwal}\\
		\\
		based on the paper by\\
		\textbf{Alexei A. Efros and William T. Freeman}
}
\date{November, 2015}
\begin{document}
\maketitle
\tableofcontents
\chapter{Introduction}
\section{Problem Statement}
The paper we have implemented seeks to solve the problem of generating unlimited amount of image data from a given sample of texture in such a way that the generated images would be perceived as the same texture.

Furthermore, the paper uses the same approach in transferring texture from one image to the other.

\section{Abstract of the Paper Implemented}
We present a simple image-based method of generating novel visual appearance in which a new image is synthesized by stitching together small patches of existing images. We call this process image quilting. First, we use quilting as a fast and very simple texture synthesis algorithm which produces surprisingly good results for a wide range of textures. Second, we extend the algorithm to perform texture transfer – rendering an object with a texture taken from a different object. More generally, we demonstrate how an image can be re-rendered in the style of a different image. The method works directly on the images and does not require 3D information.

\chapter{The Algorithm}

\chapter{Dataset Description}

\chapter{Our Implementation}

\chapter{Results}

\chapter{Conclusion}
\end{document}